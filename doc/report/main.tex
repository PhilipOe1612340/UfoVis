\documentclass{article}
\usepackage[utf8]{inputenc}

\usepackage[margin=2cm]{geometry} % adjust margin
\usepackage{paralist} %compact itemize / enumeration
\usepackage{hyperref} %\url
\usepackage{numprint} %print numbers with thousand separator

\title{UfoVis}
\author{Marius, Philip, Robin}
\date{February 2021}

\begin{document}

\maketitle

\begin{itemize}
    \item Paper Outline Guidance (5-7 pages with graphics).
    \item Introduction.  Briefly describe the problem you are solving and a high-level description of your approach.  Include your motivation for this problem.  Why is it important and why should I care? 
    \item Data set.  Describe you data set in detail, how you collected the data and any preprocessing, data cleaning, transformations, etc.\\\\
    \item Methodology
    \begin{itemize}
        \item How did you use the data?
        \item What algorithms did you use and why?
        \item How does your approach help you solve problems you identified in the introduction?
    \end{itemize}
    \item System.  Describe the essential components of your system/application.  Include a description of the users of your system and the tasks they can perform or the questions they can answer.
    \item Use Cases.  Present one or two use cases that show case the most important features of your system.
\end{itemize}


\section*{Dataset}
There are two datasets that can be used in the project to display airports, depending on the amount of airports that should be included in the database.
\begin{compactenum}[1.]
\item The first dataset is from \url{https://www.abflug.info/} and contains approximately \numprint{6500} airports. All the airports have an IATA code and an ICAO code, other airports are not included in the dataset. The dataset additionally contains an identifier, the name in english and german, the length of the largest runway in feet, a link to the wikipedia entry in german and english, and the latitude, the longitude, the elevation in feet, and the \textit{ISO 3166-1 alpha-2} (country) code.\\
Because only larger airports have an IATA code and an ICAO code, there will be no differentiation between different airport types in the visualization when using this dataset, which means you can only see whether there is an airport near an UFO sighting.
\item The second dataset is from \url{https://ourairports.com/} and contains approximately \numprint{62000} airports. The dataset additionally contains an id, the type, the latitude, longitude and elevation, the name, the continent code, the \textit{ISO 3166-1 alpha-2} code, the \textit{ISO 3166-2} (region) code, the municipality, the name of the airport, whether there is scheduled service, the gps code, the IATA code, a local code, a link to the official website of the airport, a wikipedia link to the airport, and some keywords.\\
Because of the type attribute, this dataset has a differentiation between different airport types like small airports, large airports, or balloon ports.
\end{compactenum}
Both datasets were downloaded as \textit{.csv} file. There was no data preprocessing done for them, except for replacing undefined values with null, and replacing the apostrophes such that the database can handle the values. Both datasets will also not be fully loaded into the database. Only the following important attributes are loaded into the database. The id, the name of the airport in english, the country code, the type, the coordinates and the elevation in feet.\\
%https://blog.mapbox.com/clustering-millions-of-points-on-a-map-with-supercluster-272046ec5c97
%https://github.com/Leaflet/Leaflet.markercluster
%https://d3js.org
\section*{Methodology}
Both datasets, the UFO report dataset and the airport dataset, were visualized on a map, to recognize patterns in the data. There were two attempts.\\
The first attempt was the creation of a heatmap based on the UFO data. The heatmap thereby highlighted places, where lots of UFOs were reported, but didn't lead to a recognition of patterns. Therefore a cluster map with glyphs was created.\\
This was the second attempt to discover patterns. For this visualization, parts of the UFO data and the whole airport data are loaded from the database. Each entry thereby has a marker on the geographic position of the event.
Because a lot of markers would most likely crash the system, you are not able to see whether there are makers on the same exact position, and it is hard to find patterns, a clustering algorithm was chosen to cluster events in close distance. The clustering is done by the leaflet library ''markercluster''. It uses a hierarchical greedy clustering approach. This approach selects a random marker from all the markers on the map and clusters all points within a specified distance (here: default, 80 pixel) to this marker. Further, it iteratively selects points that do not yet belong to a cluster and repeats the clustering step within the specified distance.\\
Due to the default setting of the library to show the amount of markers within a cluster, which is not practical to discover patterns, a glyph was created. Each UFO marker stores information about the shape of the UFO and each airport marker about the type of the airport. This information was used to create the glyph. The choice of the glyph icon fell on a pie chart within a donut chart, because it is easy to encode the information and best represents the center of a cluster. The inner pie chart contains the airport type, which is encoded by different inimitable colors and the outer donut chart contains the UFO shapes, also encoded by different unique colors. The decision of using the inner chart for airport types and the outer for UFO shapes, was made due to the greater amount of divers shapes than divers types, because the outer ring leaves more space to encode information.\\
\textit{This approach of using the cluster map helped to solve the problems from the introduction......}


\end{document}


